\documentclass[signature=data]{physicsreport}
\usepackage{graphicx}

%%
%% User settings
%%

\classno{}
\stuno{}
\groupno{}
\stuname{}
\expdate{\expdatefmt\today}
\expname{用惠斯通电桥测电阻}

%%
%% Document body
%%

\begin{document}
% First page
% Some titles and personal information are defined in ``\maketitle''.
\maketitle
\section{实验目的}
\section{实验预习}
\newpage

\section{实验现象及数据记录}
% Teacher signature
\makeatletter
\physicsreport@body@signature{data}
\makeatother

\newpage

% Data process and others


\section{实验结论及现象分析}
对比不能比 N 值下,惠斯通电桥灵敏度变化,并分析其他可能影响惠斯通电桥灵敏度参量


实验数据显示,惠斯通电桥的灵敏度在比例臂比值增加时先升高后降低。
当比例臂比值为1时,电桥的灵敏度达到最大。

电源电压、滑动变阻器的阻值、检流计的内阻及其自身灵敏度等因素都会影响电桥的灵敏度。
实际上,滑动变阻器的阻值与电源电压的作用相似,都是通过调整电桥两端的输入电压来调控电桥的灵敏度。

\section{讨论问题}

\subsection{电桥测电阻为什么不能测量小于 $1\Omega$ 的电阻?}

在惠斯通电桥的应用中,通常忽略线路的电阻。
然而,当待测电阻小于$1\Omega$时,电桥内部导线的电阻不再可以忽略。
此外,连接不同元件的导线电阻也各不相同,这种差异会导致惠斯通电桥在测量小电阻时准确性下降,难以得到精确的测量结果。


\subsection{用什么方法保护电流计,不至于因电流过大而损坏?}

在正式测量之前,可以先对待测电阻进行初步估算,以大致确定电阻箱的初始设定值。
在测量过程中,应从电流计的最低灵敏度开始,逐步提高灵敏度进行测量。
进行电路调整时,必须先断开电流计与电路的连接。
调整电阻箱时,最好按照从低位旋钮到高位旋钮的顺序依次进行调节。

\subsection{当电桥平衡后,若互换电源和检流计位置,电桥是否仍然平衡?并证明。}

理论上电桥仍然保持平衡。当原电桥达到平衡后,滑动变阻器位于干路中。
更换电源和检流计的位置后,滑动变阻器依然位于干路上。
通过分析电路,可以发现此时比例臂关系发生了变化,但电桥的平衡条件仍然保持不变,还是$R_xr_R=Rr_x$
根据原来的数据,新电桥在这种情况下仍能维持平衡。
\end{document}