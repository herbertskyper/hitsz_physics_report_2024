\documentclass[signature=data]{physicsreport}
\usepackage{graphicx}

%%
%% User settings
%%

\classno{}
\stuno{}
\groupno{}
\stuname{}
\expdate{\expdatefmt\today}
\expname{光电效应法测定普朗克常量}

%%
%% Document body
%%

\begin{document}
% First page
% Some titles and personal information are defined in ``\maketitle''.
\maketitle

\section{实验预习指导}
\newpage

\section{原始数据记录}
% Teacher signature
\makeatletter
\physicsreport@body@signature{data}
\makeatother

\newpage

% Data process and others
\section{数据处理}

(在三个不同直径的光阑孔下分别测量对应各个光频率 $v$ 的截止电压 $U_0$,找出两者的
线性关系。用最小二乘法与作图法求出普朗克常数 $h$ 的实验值,以及与普朗克常数标准值
$h_0 = 6.626\times 10^{-34}J \cdot s$ 的相对误差。)

\section{实验结论及现象分析}
(分析实验误差的来源,以及比较以上每种数据处理方法的优缺点)



\newpage



\section{讨论题}

\subsection{请解释什么是逸出功 $A$,以及怎样可以从截止电压 $U_0$ 与光频率 $v$ 两者的线性关系
中求出逸出功 $W$。}
\subsection{请讨论一下,不同金属材料的逸出功 $A$ 会否相同,并加以解释。}
\subsection{请讨论一下,不同金属材料的 $U_0-v$ 线性关系会否相同,并加以解释。}
\subsection{请解释什么是暗电流、本底电流、和阳极反向电流,以及它们各自出现的原因,并讨论它们各自会怎样影响“零电流法”对截止电压 $U_0$ 的测量结果。}



\end{document}