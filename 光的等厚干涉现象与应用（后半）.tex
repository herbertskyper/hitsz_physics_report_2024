\documentclass[signature=data]{physicsreport}
\usepackage{graphicx}

%%
%% User settings
%%

\classno{}
\stuno{}
\groupno{}
\stuname{}
\expdate{\expdatefmt\today}
\expname{光的等厚干涉现象与应用}

%%
%% Document body
%%

\begin{document}
% First page
% Some titles and personal information are defined in ``\maketitle''.
\maketitle

\section{实验预习指导}
\newpage

\section{原始数据记录}
% Teacher signature
\makeatletter
\physicsreport@body@signature{data}
\makeatother

\newpage

% Data process and others
\section{数据处理}
用逐差法求 $D^2_m - D^2_n$ 的平均值;计算曲率半径 $R$ 的平均值及不确定度;计算磁带的厚度(选做),要有完整的计算过程。



\newpage

\section{实验现象分析及结论}

\section{讨论题}



\subsection{理论上牛顿环中心是个暗点,实际上看到的往往是个忽明忽暗的斑,其原因是什么?对透镜曲率半径$R$测量有无影响?}

\subsection{实验中,若平板玻璃上有微小的凸起,则凸起处的干涉条纹会发生如何变化?}




\end{document}