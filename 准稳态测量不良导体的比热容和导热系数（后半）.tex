\documentclass[signature=data]{physicsreport}
\usepackage{graphicx}

%%
%% User settings
%%

\classno{}
\stuno{}
\groupno{}
\stuname{}
\expdate{\expdatefmt\today}
\expname{准稳态测量不良导体的比热容和导热系数}

%%
%% Document body
%%

\begin{document}
% First page
% Some titles and personal information are defined in ``\maketitle''.
\maketitle

\section{实验预习指导}
\newpage

\section{原始数据记录}
% Teacher signature
\makeatletter
\physicsreport@body@signature{data}
\makeatother

\newpage

% Data process and others
\section{数据处理}
\subsection{在坐标纸上分别画出 $\Delta T-\tau $ 及 $T-\tau$ 曲线,从图上判断何时进入准稳态,并求出 $\Delta T$ 及 $dT/d\tau$;}


\subsection{计算有机玻璃样品和橡胶样品的导热系数和比热容。}

\newpage

\section{实验现象分析及结论}

\section{讨论题}



\subsection{本实验中我们采取在样品两端加热的方式根据加热面与中心面的温差及端面温升速率求
出导热系数和比热。实验中为何使用四块样品?}

\subsection{本实验中判断系统进入准稳态的条件是什么?}

\subsection{本实验中准稳态会无限保持下去吗?是否时间越长实验数据越好?}


\end{document}