\documentclass[signature=data]{physicsreport}
\usepackage{graphicx}

%%
%% User settings
%%

\classno{}
\stuno{}
\groupno{}
\stuname{}
\expdate{\expdatefmt\today}
\expname{用惠斯通电桥测电阻}

%%
%% Document body
%%

\begin{document}
% First page
% Some titles and personal information are defined in ``\maketitle''.
\maketitle
\section{实验目的}
\section{实验预习}
\newpage

\section{实验现象及数据记录}
% Teacher signature
\makeatletter
\physicsreport@body@signature{data}
\makeatother

\newpage

% Data process and others


\section{实验结论及现象分析}
对比不能比 N 值下,惠斯通电桥灵敏度变化,并分析其他可能影响惠斯通电桥灵敏度参量

\section{讨论问题}

\subsection{电桥测电阻为什么不能测量小于 $1\Omega$ 的电阻?}

\subsection{用什么方法保护电流计,不至于因电流过大而损坏?}

\subsection{当电桥平衡后,若互换电源和检流计位置,电桥是否仍然平衡?并证明。}


\end{document}