\documentclass[signature=data]{physicsreport}

%%
%% User settings
%%

\classno{}
\stuno{}
\groupno{}
\stuname{}
\expdate{\expdatefmt\today}
\expname{拉伸法测杨氏弹性模量}

%%
%% Document body
%%

\begin{document}
% First page
% Some titles and personal information are defined in ``\maketitle''.
\maketitle

\section{实验预习}
\section{实验预习}

% Teacher signature
\makeatletter
\physicsreport@body@signature{preparation}
\makeatother

\newpage

% Original experiment data
\section{实验现象及原始数据记录}


% Teacher signature
\makeatletter
\physicsreport@body@signature{data}
\makeatother

\newpage
% Data process and others
\section{数据处理}
(要有详细的计算过程,推导不确定度的表达式,计算杨氏模量及其不确定度,给出完整的测量结果表达形式)

由于L,H,D测量误差满足均匀分布,故不确定度有:

\begin{gather*}
    U_L = \frac{0.8}{\sqrt 3}mm = 0.47mm\\
    E_L = \frac{U_L}{L} = 0.064\%\\
    P = 68.3\%\\
    U_H = \frac{0.8}{\sqrt 3}mm = 0.47mm\\
    E_H = \frac{U_H}{H} = 0.068\%\\
    P = 68.3\%\\
    U_D = \frac{0.02}{\sqrt 3}mm = 0.012mm\\
    E_D = \frac{U_D}{D} = 0.030\%\\
    P = 68.3\%\\
\end{gather*}

金属丝测量误差满足均匀分布,既有A类不确定度也有B类不确定度:

\begin{gather*}
    \bar d = \frac{\sum_{i=1}^{6}d_i}{6} = 0.598mm\\
    U_{dA} = \sqrt{ \frac{\sum_{i=1}^{6}(d_i - \bar d)^2}{6 \times (6-1)} } = 0.0012mm\\
    U_{dB} = \frac{0.004}{\sqrt 3}mm = 0.0023mm\\
    U_d = \sqrt{U_{dA}^2 + U_{dB}^2} = 0.0026mm\\
    E_d = \frac{U_d}{\bar d} = 0.44\%\\
    P = 68.3\%
\end{gather*}

标尺刻度$\Delta x$既有A类不确定度也有B类不确定度:

\begin{gather*}
    \bar{\Delta x} = \frac{\sum_{i=1}^{5}\Delta x_i}{5} = 21.86mm\\
    U_{\Delta x A} = \sqrt{ \frac{\sum_{i=1}^{5}(\Delta x_i - \bar{\Delta x})^2}{5 \times (5-1)} } = 0.130mm\\
    U_{\Delta x B} = \sqrt{2} \frac{0.5}{\sqrt 3} = 0.408mm\\
    U_{\Delta x} = \sqrt{ U_{\Delta x A}^2 + U_{\Delta x B}^2 } = 0.428mm\\
    E_{\Delta x} = \frac{U_{\Delta x}}{\bar{\Delta x}} = 1.96\%\\
\end{gather*}

对杨氏模量计算公式两边取对数,有:

\begin{gather*}
    \ln E = \ln \frac{8 \Delta f}{\pi} + \ln L + \ln H - 2 \ln d - \ln D - \ln \Delta x
\end{gather*}

杨氏模量的相对不确定度推导如下:

\begin{align*}
    E_u &= \sqrt{(U_L \frac{\partial \ln E}{\partial L})^2 + (U_H \frac{\partial \ln E}{\partial d})^2 + (U_d \frac{\partial \ln E}{\partial d})^2 + (U_D \frac{\partial \ln E}{\partial D})^2 + (U_{\Delta x} \frac{\partial \ln E }{\partial \Delta x})^2} \\
        &= \sqrt{ (\frac{U_L}{L})^2 + (\frac{U_H}{H})^2 + (\frac{2U_d}{d})^2 + (\frac{U_D}{D})^2 + (\frac{U_{\Delta x}}{\Delta x})^2 } \\
        &= \sqrt{ E_L^2 + E_H^2 + 4E_d^2 + E_D^2 + E_{\Delta x}^2 } \
\end{align*}

故计算相对不确定度得:

\begin{align*}
         E_u &= \sqrt{ E_L^2 + E_H^2 + 4E_d^2 + E_D^2 + E_{\Delta x}^2 } \\
             &= 2.2\%\\
E_u \times E &= 0.04 \times 10^{11}Pa
\end{align*}

最终杨氏模量测量结果:

\begin{gather*}
E = (2.02 \pm 0.04) \times 10^{11}Pa\\
E_u = 2.2\%\\
P = 68.3\%
\end{gather*}

\section{实验结论及误差分析}
本实验通过拉伸法测量了钢丝的杨氏模量,得到的结果为$(2.02 \pm 0.04) \times 10^{11}Pa$,相对不确定度约为2.2\%,符合实验要求。

误差分析:本实验的误差主要来自于测量仪器的误差,如千分尺、游标卡尺、钢尺的刻度误差。此外,实验过程中,金属丝的形变、摩擦、加力减力是否均匀等因素也会引入误差。

本实验通过对数据的逐差法处理,消除了一部分数据偏差,提高了实验结果的准确性。

为了进一步减小误差,可以提高测量仪器的精度,减小测量误差,同时在实验过程中尽量减小外界因素的影响。

\section{讨论题}
\begin{enumerate}
    \item 材料相同,但粗细、长度不同的两根钢丝,它们的杨氏模量是否相同?
 
    相同。杨氏模量只与材料的物理性质有关,与材料的大小、形状无关。

    \item 从误差分析的角度分析为什么同是长度测量,需要采用不同的量具?
    
    对于金属丝直径$d$等较小的长度量精确度要求较高,故采用千分尺高精度测量。而对于较大长度,金属丝的原长和反射镜转轴到标尺的垂直距离$L,H$,精度要求较低,故采用钢尺测量。
    而光杠杆常数$D$介于二者之间,用游标卡尺测量。

    \item 实验过程中为什么加力和减力过程,施力螺母不能回旋?
    
    回旋施力螺母可能会导致金属丝来不及发生形变,也可能引入额外的摩擦,从而增加了实验中的不确定性和误差。

    \item 用逐差法处理数据的优点是什么?应该注意什么问题?
    逐差法可以充分利用所有数据,消除一部分由于系统误差引起的数据偏差。

    注意问题:逐差法的有效性取决于原始数据的质量。如果原始数据的误差较大,逐差法会放大误差,得到的结果不准确。
\end{enumerate}

\end{document}