\documentclass[signature=data]{physicsreport}
\usepackage{graphicx}

%%
%% User settings
%%

\classno{}
\stuno{}
\groupno{}
\stuname{}
\expdate{\expdatefmt\today}
\expname{光的等厚干涉现象与应用}

%%
%% Document body
%%

\begin{document}
% First page
% Some titles and personal information are defined in ``\maketitle''.
\maketitle

\section{实验预习指导}
\newpage

\section{原始数据记录}
% Teacher signature
\makeatletter
\physicsreport@body@signature{data}
\makeatother

\newpage

% Data process and others
\section{数据处理}

用逐差法求 $D^2_m - D^2_n$ 的平均值;计算曲率半径 $R$ 的平均值及不确定度;计算磁带的厚度(选做),要有完整的计算过程。

\subsection{牛顿环实验}

根据逐差法, 得 

$D^2_m - D^2_n$ 的平均值 $ \overline{D_{m}^{2}-D_{n}^{2}}=\frac{1}{11} \sum_{n=10}^{2}\left(D_{n+11}^{2}-D_{n}^{2}\right)=22.98 \mathrm{~mm}^{2} $ 。

曲率半径的平均值  $\bar{R}=\frac{\overline{D_{m}^{2}-D_{n}^{2}}}{4(m-n) \lambda}=\frac{22.98 \mathrm{~mm}^{2}}{44 \times 589.3 \mathrm{~nm}}=886.15 \mathrm{~mm} $ 。

计算不确定度的公式如下:

$U_{\overline{D^2_m-D^2_n}} = \sqrt {S^2_{D^2_m - D^2_n} + u^2} \approx S^2_{D^2_m - D^2_n} = \sqrt{\frac{1}{11 \times 10} \sum_{i=1}^{11} [(D^2_m - D^2_{n}) - \overline{(D^2_{m} - D^2_{n})} ]^2}$

代入数据计算得$U_{\overline{D^2_m-D^2_n}}=0.02596\mathrm{~mm}^{2}$

不确定度$U_{\overline{R}}=\frac{U_{\overline{D^2_m-D^2_n}}}{4(m-n)\lambda}=\frac{0.02596 \mathrm{~mm}^{2}}{44 \times 589.3 \mathrm{~nm}}=1.001\mathrm{mm}$

\subsection{磁带厚度测量}
劈棱到磁带端的总长$\overline{L}=18.420mm$

通过实验数据计算得到$\overline{l}=6.066mm$

则单位长度干涉条纹条数为$\frac{10}{\overline{l}}=1.649mm^{-1}$

磁带厚度$d=\frac{h\overline{L}\lambda}{2}=8.950\mu m $



\newpage

\section{实验现象分析及结论}
干涉条纹等距,中心会看到忽明忽暗的斑。

计算得到的数据如上,相对误差较小,与实际数据有一定差异,可能因为操作不当,测量不准确等原因。

\section{讨论题}



\subsection{理论上牛顿环中心是个暗点,实际上看到的往往是个忽明忽暗的斑,其原因是什么?对透镜曲率半径$R$测量有无影响?}

由于透镜和玻璃板表面不完美,存在微小的不规则性,导致空气薄膜厚度不均匀。此外,光源的非单色性也会造成干涉条纹不清晰,从而使中心点出现忽明忽暗的现象。

这种现象对透镜曲率半径$R$的测量影响较小,因为牛顿环的半径仍然可以通过外部较清晰的干涉环准确测量。不过,为了提高精度,最好使用单色光并尽量减少表面缺陷的影响。

\subsection{实验中,若平板玻璃上有微小的凸起,则凸起处的干涉条纹会发生如何变化?}

凸起处的干涉条纹会向外弯曲。

\end{document}