\documentclass[signature=data]{physicsreport}

%%
%% User settings
%%

\classno{}
\stuno{}
\groupno{}
\stuname{}
\expdate{\expdatefmt\today}
\expname{拉伸法测杨氏弹性模量}

%%
%% Document body
%%

\begin{document}
% First page
% Some titles and personal information are defined in ``\maketitle''.
\maketitle

\section{实验预习}
\section{实验预习}

% Teacher signature
\makeatletter
\physicsreport@body@signature{preparation}
\makeatother

\newpage

% Original experiment data
\section{实验现象及原始数据记录}


% Teacher signature
\makeatletter
\physicsreport@body@signature{data}
\makeatother

\newpage
% Data process and others
\section{数据处理}
(要有详细的计算过程,推导不确定度的表达式,计算杨氏模量及其不确定度,给出完整的测量结果表达形式)
\vspace{12em}
\section{实验结论及误差分析}

\section{讨论题}
\begin{enumerate}
    \item 材料相同,但粗细、长度不同的两根钢丝,它们的杨氏模量是否相同?
 

    \item 从误差分析的角度分析为什么同是长度测量,需要采用不同的量具?
    

    \item 实验过程中为什么加力和减力过程,施力螺母不能回旋?
    

    \item 用逐差法处理数据的优点是什么?应该注意什么问题?
\end{enumerate}

\end{document}